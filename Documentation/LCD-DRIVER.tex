\documentclass{article}

%
% Import packages
% ===============

\usepackage[T1]{fontenc}
\usepackage{bytefield}
\usepackage{multicol}
\usepackage{geometry}
\usepackage{lipsum}
\usepackage{float}
\usepackage{parskip}
\usepackage[hidelinks]{hyperref}
\usepackage{xcolor}

%
% Change links style
%===================
\hypersetup{
	colorlinks=true,
	linkcolor={red!50!black},
	urlcolor={blue!80!black}
}

%
% Change default font
% ===================
\renewcommand{\familydefault}{\sfdefault}

%
% Setup geometry
%===============
\geometry{
	a4paper,
	total={170mm,257mm},
	left=20mm,
	top=20mm,
}

\newcommand{\regs}[2]{
	\begin{tabular}{l l}
		Offset: & \texttt{#1} \\
		Length: & \texttt{#2}
	\end{tabular} \\
	}

\title{LCD-DRIVER}
\author{Olimex Ltd.}
\begin{document}
	\maketitle
	\tableofcontents
	\listoffigures
	\pagebreak


	\section{Description}
	\begin{multicols}{2}
		\lipsum
	\end{multicols}

	\section{EEPROM content}
	
	Memory layout can be seen on \ref{fig:EEPROM_LAYOUT}. It's separated into several sections.
	Most of them are fixed length, except configuration.

	\begin{figure}[H]
		\centering
		\begin{bytefield}[bitheight=4\baselineskip]{32}
	\begin{rightwordgroup}{4 bytes}
		\bitbox[]{4}{
			\texttt{0x00} \\[2\baselineskip]
			\texttt{0x03}
			} &
		\bitbox{28}{Header}
	\end{rightwordgroup} \\
	\begin{rightwordgroup}{4 bytes}
		\bitbox[]{4}{
			\texttt{0x04} \\[2\baselineskip]
			\texttt{0x07}
			} &
		\bitbox{28}{ID}
	\end{rightwordgroup} \\
	\begin{rightwordgroup}{4 bytes}
		\bitbox[]{4}{
			\texttt{0x08} \\[2\baselineskip]
			\texttt{0x0B}
			} &
		\bitbox{28}{Revision}
	\end{rightwordgroup} \\
	\begin{rightwordgroup}{4 bytes}
		\bitbox[]{4}{
			\texttt{0x0C} \\[2\baselineskip]
			\texttt{0x0F}
			} &
		\bitbox{28}{Serial}
	\end{rightwordgroup} \\
	\begin{rightwordgroup}{164 bytes}
		\bitbox[]{4}{
			\texttt{0x10} \\[2\baselineskip]
			\texttt{0xB3}
			} &
		\bitbox{28}{Configuration}
	\end{rightwordgroup} \\
	\begin{rightwordgroup}{72 bytes}
		\bitbox[]{4}{
			\texttt{0xB4} \\[2\baselineskip]
			\texttt{0xFB}
			} &
		\bitbox{28}{Reserved}
	\end{rightwordgroup} \\
	\begin{rightwordgroup}{4 bytes}
		\bitbox[]{4}{
			\texttt{0xFC} \\[2\baselineskip]
			\texttt{0xFF}
			} &
		\bitbox{28}{Checksum}
	\end{rightwordgroup} \\
\end{bytefield}

		\caption{EEPROM layout}
		\label{fig:EEPROM_LAYOUT}
	\end{figure}

	\subsection{Header}
	\regs{0x00}{0x04}

	The header is used to identify beginning of LCD-OLinuXino configuration. The value must be
	\texttt{0x4F4CB727}.

	\subsection{ID}
	\regs{0x04}{0x04}

	This is unique value for each LCD used. Possibilities are:
	\begin{itemize}
		\item \texttt{7839} - LCD-OLinuXino-10 \\
		\url{https://www.olimex.com/Products/OLinuXino/LCD/LCD-OLinuXino-10/open-source-hardware}

		\item \texttt{7864} - LCD-OLinuXino-7 \\
		\url{https://www.olimex.com/Products/OLinuXino/LCD/LCD-OLinuXino-7/open-source-hardware}

		\item \texttt{8630} - LCD-OLinuXino-5 \\
		\url{https://www.olimex.com/Products/OLinuXino/LCD/LCD-OLinuXino-5/open-source-hardware}

		\item \texttt{7859} - LCD-OLinuXino-4.3 \\
		\url{https://www.olimex.com/Products/OLinuXino/LCD/LCD-OLinuXino-4.3TS/open-source-hardware}
	\end{itemize}

	\subsection{Revision}
	\regs{0x08}{0x04}

	This field represent board hardware revision.

	\subsection{Serial}
	\regs{0x0C}{0x04}

	Unique serial number for each board.
	
	

	\subsection{Configuration}
	\regs{0x10}{--------}

	Configuration section holds information about the timings and the LCD itself.
	The layout is shown at Figure \ref{fig:EEPROM_CONFIG}.
	
	\begin{figure}[H]
		\centering
		\begin{bytefield}[
	leftcurly=.,
	leftcurlyspace=0pt,
	bitformatting={\small\ttfamily},
	boxformatting={\centering\small},
	endianness=big]{32}
	\begin{leftwordgroup}{\texttt{0x10} & \\ \\ \\ \\ \\ & \texttt{0x43}}
		\wordbox{4}{Info}
	\end{leftwordgroup} \\
	\begin{leftwordgroup}{\texttt{0x44} & \\[16em] & \texttt{0x6F}}
		\wordbox{11}{Mode}
	\end{leftwordgroup}
\end{bytefield}

		\caption{Timings section}
		\label{fig:EEPROM_CONFIG}
	\end{figure}
	
	The section can be divided to two smaller subsections:
	\begin{itemize}
		\item Info
		\item Mode
	\end{itemize}

		\subsubsection{Info}
		The layout of this section is shown at Figure \ref{fig:EEPROM_INFO}.
		\begin{figure}[H]
			\centering
			\begin{bytefield}[
	leftcurly=.,
	leftcurlyspace=0pt,
	bitformatting={\small\ttfamily},
	boxformatting={\centering\small},
	endianness=big]{32}
	\begin{leftwordgroup}{\texttt{0x10} & \\ \\ \\ \\ \\ \\ \\ \\ \\ \\ & \texttt{0x2C}}
		\wordbox{8}{Name}
	\end{leftwordgroup} \\
	\begin{leftwordgroup}{\texttt{0x30}}
		\wordbox{1}{Bits per color channel}
	\end{leftwordgroup} \\
	\begin{leftwordgroup}{\texttt{0x34}}
		\wordbox{1}{Width}
	\end{leftwordgroup} \\
	\begin{leftwordgroup}{\texttt{0x38}}
		\wordbox{1}{Height}
	\end{leftwordgroup} \\
	\begin{leftwordgroup}{\texttt{0x3C}}
		\wordbox{1}{Bus format}
	\end{leftwordgroup} \\
	\begin{leftwordgroup}{\texttt{0x40}}
		\wordbox{1}{Bus flags}
	\end{leftwordgroup}
\end{bytefield}

			\caption{Info section}
			\label{fig:EEPROM_INFO}
		\end{figure}
		
		This field contains the following fields.
		\begin{itemize}
			\item Name -- The name of the board, e.g. \texttt{"LCD-OLinuXino-7}
			\item Bits per color channel -- Number of bits describing one color, typically 8
			\item Width -- Physical width of the panel in millimeters
			\item Height -- Physical height of the panel in millimeters
			\item Bus format -- The value must be get from
			\textbf{include/uapi/linux/media-bus-format.h}
			\item Bus flags -- The value must be get from \textbf{include/uapi/drm/drm\_mode.h}
		\end{itemize}
		
			

		\subsubsection{Mode}
		The fields in this subsection describes timing requirements of the LCD. The layout of this section is shown at Figure \ref{fig:EEPROM_MODES}.
		\begin{figure}[H]
			\centering
			\begin{bytefield}[
	leftcurly=.,
	leftcurlyspace=0pt,
	bitformatting={\small\ttfamily},
	boxformatting={\centering\small},
	endianness=big]{32}
	
	
	\begin{leftwordgroup}{\texttt{0x44}}
		\wordbox{1}{Modes number}
	\end{leftwordgroup} \\
	\\
	\begin{leftwordgroup}{\texttt{0x48}}
		\wordbox{1}{Pixelclock}
	\end{leftwordgroup} \\
	\begin{leftwordgroup}{\texttt{0x4C}}
		\wordbox{1}{H. Active}
	\end{leftwordgroup} \\
	\begin{leftwordgroup}{\texttt{0x50}}
		\wordbox{1}{H. Front porch}
	\end{leftwordgroup} \\
	\begin{leftwordgroup}{\texttt{0x54}}
		\wordbox{1}{H. Back porch}
	\end{leftwordgroup} \\
	\begin{leftwordgroup}{\texttt{0x58}}
		\wordbox{1}{H. Pulse width}
	\end{leftwordgroup} \\
	\begin{leftwordgroup}{\texttt{0x5C}}
		\wordbox{1}{V. Active}
	\end{leftwordgroup} \\
	\begin{leftwordgroup}{\texttt{0x60}}
		\wordbox{1}{V. Front porch}
	\end{leftwordgroup} \\
	\begin{leftwordgroup}{\texttt{0x64}}
		\wordbox{1}{V. Back porch}
	\end{leftwordgroup} \\
	\begin{leftwordgroup}{\texttt{0x68}}
		\wordbox{1}{V. Pulse width}
	\end{leftwordgroup}	\\
	\begin{leftwordgroup}{\texttt{0x6C}}
		\wordbox{1}{Refresh rate}
	\end{leftwordgroup} \\
	\begin{leftwordgroup}{\texttt{0x70}}
		\wordbox{1}{Timing flags}
	\end{leftwordgroup}	
\end{bytefield}

			\caption{Modes section}
			\label{fig:EEPROM_MODES}
		\end{figure}
	
		The fields are:
		\begin{itemize}
			\item Modes number -- The total number of modes stored
			\item Pixelclock -- Frequency of the pixel-clock in \textbf{kHz}
			\item Horizontal active area
			\item Horizontal front porch
			\item Horizontal back porch
			\item Horizontal pulse width
			\item Vertical active area
			\item Vertical front porch
			\item Vertical back porch
			\item Vertical pulse width
			\item Refresh rate
			\item Timing flags
		\end{itemize}
		
		

		
	\subsection{Checksum}
	\regs{0xFC}{0x04}
	
	The checksum is used to verify data integrity. It's calculated as CRC32.

\end{document}
