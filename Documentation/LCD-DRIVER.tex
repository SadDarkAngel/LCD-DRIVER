\documentclass{article}

%
% Import packages
% ===============

\usepackage[T1]{fontenc}
\usepackage{bytefield}
\usepackage{multicol}
\usepackage{geometry}
\usepackage{lipsum}
\usepackage{float}
\usepackage{parskip}
\usepackage[hidelinks]{hyperref}
\usepackage{xcolor}

%
% Change links style
%===================
\hypersetup{
	colorlinks=true,
	linkcolor={red!50!black},
	urlcolor={blue!80!black}
}

%
% Change default font
% ===================
\renewcommand{\familydefault}{\sfdefault}


%
% Setup geometry
%===============
\geometry{
	a4paper,
	total={170mm,257mm},
	left=20mm,
	top=20mm,
}

\newcommand{\regs}[2]{
	\begin{tabular}{l l}
		Offset: & \texttt{#1} \\
		Length: & \texttt{#2}
	\end{tabular} \\
	}


\title{LCD-DRIVER}
\author{Olimex Ltd.}
\begin{document}
	\maketitle
	\tableofcontents
	\listoffigures
	\pagebreak


	\section{Description}
	\begin{multicols}{2}
		\lipsum
	\end{multicols}

	\section{EEPROM content}

	\begin{figure}[H]
		\centering
		\begin{bytefield}[bitheight=4\baselineskip]{32}
	\begin{rightwordgroup}{4 bytes}
		\bitbox[]{4}{
			\texttt{0x00} \\[2\baselineskip]
			\texttt{0x03}
			} &
		\bitbox{28}{Header}
	\end{rightwordgroup} \\
	\begin{rightwordgroup}{4 bytes}
		\bitbox[]{4}{
			\texttt{0x04} \\[2\baselineskip]
			\texttt{0x07}
			} &
		\bitbox{28}{ID}
	\end{rightwordgroup} \\
	\begin{rightwordgroup}{4 bytes}
		\bitbox[]{4}{
			\texttt{0x08} \\[2\baselineskip]
			\texttt{0x0B}
			} &
		\bitbox{28}{Revision}
	\end{rightwordgroup} \\
	\begin{rightwordgroup}{4 bytes}
		\bitbox[]{4}{
			\texttt{0x0C} \\[2\baselineskip]
			\texttt{0x0F}
			} &
		\bitbox{28}{Serial}
	\end{rightwordgroup} \\
	\begin{rightwordgroup}{164 bytes}
		\bitbox[]{4}{
			\texttt{0x10} \\[2\baselineskip]
			\texttt{0xB3}
			} &
		\bitbox{28}{Configuration}
	\end{rightwordgroup} \\
	\begin{rightwordgroup}{72 bytes}
		\bitbox[]{4}{
			\texttt{0xB4} \\[2\baselineskip]
			\texttt{0xFB}
			} &
		\bitbox{28}{Reserved}
	\end{rightwordgroup} \\
	\begin{rightwordgroup}{4 bytes}
		\bitbox[]{4}{
			\texttt{0xFC} \\[2\baselineskip]
			\texttt{0xFF}
			} &
		\bitbox{28}{Checksum}
	\end{rightwordgroup} \\
\end{bytefield}

		\caption{EEPROM layout}
		\label{fig:EEPROM_LAYOUT}
	\end{figure}

	\subsection{Header}
	\regs{0x00}{0x04}

	The header is used to identify beginning of LCD-OLinuXino configuration. The value must be
	\texttt{0x4F4CB727}.

	\subsection{ID}
	\regs{0x04}{0x04}

	This is unique value for each LCD used. Possibilities are:
	\begin{itemize}
		\item \texttt{7839} - LCD-OLinuXino-10 \\
		\url{https://www.olimex.com/Products/OLinuXino/LCD/LCD-OLinuXino-10/open-source-hardware}

		\item \texttt{7864} - LCD-OLinuXino-7 \\
		\url{https://www.olimex.com/Products/OLinuXino/LCD/LCD-OLinuXino-7/open-source-hardware}

		\item \texttt{8630} - LCD-OLinuXino-5 \\
		\url{https://www.olimex.com/Products/OLinuXino/LCD/LCD-OLinuXino-5/open-source-hardware}

		\item \texttt{7859} - LCD-OLinuXino-4.3 \\
		\url{https://www.olimex.com/Products/OLinuXino/LCD/LCD-OLinuXino-4.3TS/open-source-hardware}
	\end{itemize}

	\subsection{Revision}
	\regs{0x08}{0x04}

	This field represent board hardware revision.

	\subsection{Serial}
	\regs{0x0C}{0x04}

	Unique serial number for each board.

	\subsection{Configuration}
	\regs{0x10}{0xA4}


	Configuration section holds information about the timings. Once read the controller must
	determine if they are suitable. The section can be divided to two smaller subsections:
	\begin{itemize}
		\item Timing
		\item Info
	\end{itemize}

	The layout is shown at Figure \ref{figure:Configuration}.

		\subsubsection{Timing}
		The fields in this subsection describes timing requirements of the LCD. Each field is 3 bytes long
		representing minimal, typical and maximum value. \\
		The fields are:
		\begin{itemize}
			\item Pixelclock
			\item Horizontal active area
			\item Horizontal front porch
			\item Horizontal back porch
			\item Horizontal pulse width
			\item Vertical active area
			\item Vertical front porch
			\item Vertical back porch
			\item Vertical pulse width
			\item Timing flags
		\end{itemize}

		\begin{figure}
			\centering
			\begin{bytefield}[
	leftcurly=.,
	leftcurlyspace=0pt,
	bitformatting={\small\ttfamily},
	boxformatting={\centering\small},
	endianness=big]{32}
	\begin{rightwordgroup}{Timing}
		\begin{leftwordgroup}{\texttt{0x10} & \\ \\ \\ & \texttt{0x18}}
			\wordbox{3}{Pixelclock}
		\end{leftwordgroup} \\
		\begin{leftwordgroup}{\texttt{0x1C} & \\ \\ \\  & \texttt{0x24}}
			\wordbox{3}{H. Active}
		\end{leftwordgroup} \\
		\begin{leftwordgroup}{\texttt{0x28} & \\ \\ \\  & \texttt{0x30}}
			\wordbox{3}{H. Front porch}
		\end{leftwordgroup} \\
		\begin{leftwordgroup}{\texttt{0x34} & \\ \\ \\  & \texttt{0x3C}}
			\wordbox{3}{H. Back porch}
		\end{leftwordgroup} \\
		\begin{leftwordgroup}{\texttt{0x40} & \\ \\ \\  & \texttt{0x48}}
			\wordbox{3}{H. Pulse width}
		\end{leftwordgroup} \\
		\begin{leftwordgroup}{\texttt{0x4C} & \\ \\ \\  & \texttt{0x54}}
			\wordbox{3}{V. Active}
		\end{leftwordgroup} \\
		\begin{leftwordgroup}{\texttt{0x58} & \\ \\ \\  & \texttt{0x60}}
			\wordbox{3}{V. Front porch}
		\end{leftwordgroup} \\
		\begin{leftwordgroup}{\texttt{0x64} & \\ \\ \\  & \texttt{0x6C}}
			\wordbox{3}{V. Back porch}
		\end{leftwordgroup} \\
		\begin{leftwordgroup}{\texttt{0x70} & \\ \\ \\  & \texttt{0x78}}
			\wordbox{3}{V. Pulse width}
		\end{leftwordgroup}	\\
		\begin{leftwordgroup}{\texttt{0x7C}}
			\wordbox{1}{Timing flags}
		\end{leftwordgroup}
	\end{rightwordgroup} \\
	\\
	\begin{rightwordgroup}{Info}
		\begin{leftwordgroup}{\texttt{0x80} & \\ \\ \\ \\ \\ \\ \\ \\ \\ \\ & \texttt{0x9C}}
			\wordbox{8}{Name}
		\end{leftwordgroup} \\
		\begin{leftwordgroup}{\texttt{0xA0}}
			\wordbox{1}{Bits per color channel}
		\end{leftwordgroup} \\
		\begin{leftwordgroup}{\texttt{0xA4}}
			\wordbox{1}{Width}
		\end{leftwordgroup} \\
		\begin{leftwordgroup}{\texttt{0xA8}}
			\wordbox{1}{Height}
		\end{leftwordgroup} \\
		\begin{leftwordgroup}{\texttt{0xAC}}
			\wordbox{1}{Bus format}
		\end{leftwordgroup} \\
		\begin{leftwordgroup}{\texttt{0xB0}}
			\wordbox{1}{Bus flags}
		\end{leftwordgroup}
	\end{rightwordgroup}
\end{bytefield}

			\label{figure:Configuration}
			\caption{Timings section}
		\end{figure}

		\subsubsection{Info}
		This field contains additional fields.
		\begin{itemize}
			\item Name
			\item Bits per color channel
			\item Width
			\item Height
			\item Bus format
			\item Bus flags
		\end{itemize}

\end{document}
